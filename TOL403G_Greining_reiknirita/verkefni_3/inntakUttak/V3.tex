\documentclass[a4paper]{article}

\usepackage[icelandic]{babel}
\usepackage[framed]{mcode}
\usepackage{listingsutf8}

\usepackage{t1enc}
\usepackage{multicol}
\usepackage{amsmath}
\usepackage{amssymb}
\usepackage{tikz}
\usepackage[T1]{fontenc}
\usepackage{amsmath}
\usepackage{graphicx}
\usepackage[colorinlistoftodos]{todonotes}

\begin{document}

\title{
  TÖL403G Greining Reiknirita \\
  Verkefni 3 }

\author{
\textbf{Nemendur:}\\
  Daníel Eldjárn Vilhjálmsson \\
  Einar Smári Einarsson       \\
  Kjartan Traustason \\ \\
\textbf{Kennari:}\\
Páll Melsteð
}

\maketitle
Athygli er vakin á því að útreiknigar á simple.in gefa annað spantré en lausn kennarans. Sú lausn er þó engu að síður rétt.\\ ö\

Því miður tókst okkur ekki að útfæra endanlegu hugmyndina okkar varðandi hvernig mætti leysa verkefnið án þess að fara í gegnum n-1 MST verkefni. Hugmyndin okkar var þó eftirfarandi:\\ \\

Ef við fjarlægjum legg, e, úr netinu okkar, G = \{V, E\},  sem er hluti af minnsta span trénu fyrir netið, brotnar upprunalega minnsta span tréið upp í tvo hluta, A og B. Til þess að finna minnsta span tréð fyrir þetta nýja net nægir okkur að finna þann legg með minnsta vægi sem tengir saman hlutana A og B. Tré sem inniheldur þennan legg verður þá minnsta span tré netsins G\_1 = \{V, E - \{e\}\}. \\ \\

Fyrsta skrefið í þessari lausn er að finna út hvaða hnútar tilheyra hópi A og hvaða hópar tilheyra hnúti B. Til þess datt okkur í hug að nota DFS (e. Depth-First Search) á einhvern hnút og finna þannig út alla hnúta sem tengjast þessum gefna hnúti. Þeir hnútar væru í hópi A. Allir hnútar sem tengjast þessum hnúti ekki í gegnum legg sem var hluti af upprunalega MSTinu tilheyra því hópi B.\\ \\

Síðan væri hægt að finna þann legg með minnst vægi sem tengir þessa tvo leggi saman og setja hann í hið nýja span tré. \\

\textbf{Keyrsla forrits:} python prim.py inntak


\end{document}